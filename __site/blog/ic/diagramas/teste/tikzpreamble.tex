\usepackage{amsmath}
\usepackage{amsthm}
\usepackage{mathtools}
\usepackage{amssymb}
\usepackage{extarrows}
\usepackage{xcolor-material}


%%%%%%%%%%%%% TIKZ %%%%%%%%%%%%%%%%%%

\usepackage{tikz}
\usepackage{tikz-cd}
\usepackage{pgfplots}
\usetikzlibrary{arrows,decorations.markings}
\usetikzlibrary{shapes.geometric}
\usetikzlibrary{babel,positioning,calc,fit,math}

\makeatletter
\newcommand*\mysize{%
   \@setfontsize\mysize{7}{9.0}%
}
\makeatother

\tikzset{
  oes/.style={
    decorate,
    decoration={
      show path construction,
      moveto code={},
      lineto code={
        \path [#1]
        (\tikzinputsegmentfirst) -- (\tikzinputsegmentlast);},
      curveto code={
        \path [#1] 
        (\tikzinputsegmentfirst) .. controls
        (\tikzinputsegmentsupporta) and (\tikzinputsegmentsupportb)
        .. (\tikzinputsegmentlast);},
      closepath code={
        \path [#1]
        (\tikzinputsegmentfirst) -- (\tikzinputsegmentlast);},},
  },
  between/.style args={#1 and #2}{
    anchor=base, yshift=-1mm,
    at = ($(#1)!0.5!(#2)$)
  },
  vs/.style={
    left=.8mm,font=\mysize
  },
  box/.style={
    minimum size=.6cm,draw
  },
  tbox/.style={
    minimum size=.6cm,draw,
    trapezium,trapezium left angle=80,trapezium right angle=0
  },
  ttbox/.style={
    minimum size=.6cm,draw,
    trapezium,trapezium left angle=90,trapezium right angle=-80
  },
  vec/.style={
    circle,
    draw,
    minimum size=.5cm,
    inner sep=0.1mm
  },
  bv/.style={
    regular polygon, regular polygon sides = 3,
    draw, font=\scriptsize,
    minimum size=0.4,
    inner sep=0.2mm
  },
  bf/.style={
    regular polygon, regular polygon sides = 3,
    draw, font=\scriptsize,
    shape border rotate=180,
    minimum size=0.4,
    inner sep=0.2mm
  },
  ->-/.style={draw,
    decoration={
    markings,
    mark=at position .6 with {\arrow{>[scale=1,width=4]}}},
    postaction={decorate}
  },
  ->-l/.style={draw,
  decoration={
  markings,
  mark=at position .4 with {\arrow{>[scale=1,width=4]}}},
  postaction={decorate}
  },
  -<-l/.style={draw,
  decoration={
  markings,
  mark=at position .7 with {\arrow{<[scale=1,width=4]}}},
  postaction={decorate}
  },
  disp/.style={
    node distance=0cm, 
    font=\normalsize,
    thick,
    baseline=-0.65ex
  },
  rofeq/.style={
    right=.25 of eq,
  },
  matd/.style={
    matrix of nodes,
    row sep=.7cm,
    column sep=.7cm,
    every node/.style={coordinate},
    nodes in empty cells
  }
}
\newcommand{\eq}[2][.2]{\node[right= #1 of #2] (eq) {$=$}}
\newcommand{\defeq}[2][.2]{\node[right= #1 of #2] (eq) {$:=$}}

\newcommand{\tails}[3][1]{%
\foreach \i in {1,...,#1} {
  \def\x{\i-#1/2-1/2}
  \def\a{-50*\i + 50*#1/2 + 50/2 - 90}
  \coordinate (#2-t-\i) at ($(#2.south)-(\x,.45)$);
  \ifnum \i > #3
    \draw[->-] (#2-t-\i) to[out=90,in=\a]
    coordinate (#2-ts-\i) (#2);
  \else
    \draw[->-] (#2) to[out=\a,in=90]
    coordinate (#2-ts-\i) (#2-t-\i);
  \fi
}}
\newcommand{\heads}[3][1]{%
\foreach \i in {1,...,#1} {
  \def\x{\i-#1/2-1/2}
  \def\a{-50*\i + 50*#1/2 + 50/2 + 90}
  \coordinate (#2-h-\i) at ($(#2.north)+(\x,.45)$);
  \ifnum \i > #3
    \draw[->-] (#2-h-\i) to[out=-90,in=\a] 
    coordinate (#2-hs-\i) (#2);
  \else
    \draw[->-] (#2) to[out=\a,in=-90]
    coordinate (#2-hs-\i) (#2-h-\i);
  \fi
}}
\newcommand{\ftails}[3][1]{%
\foreach \i in {1,...,#1} {
  \def\x{-\i+#1/2+1/2}
  \def\a{-50*\i + 50*#1/2 + 50/2 - 90}
  \coordinate (#2-t-\i) at ($(#2.south)-(\x,.45)$);
  \ifnum \i > #3
    \draw[->-] (#2-t-\i) to[out=90,in=\a]
    coordinate (#2-ts-\i) (#2);
  \else
    \draw[->-] (#2) to[out=\a,in=90]
    coordinate (#2-ts-\i) (#2-t-\i);
  \fi
}}
\newcommand{\fheads}[3][1]{%
\foreach \i in {1,...,#1} {
  \def\x{-\i+#1/2+1/2}
  \def\a{50*\i - 50*#1/2 - 50/2 + 90}
  \coordinate (#2-h-\i) at ($(#2.north)+(\x,.45)$);
  \ifnum \i > #3
    \draw[->-] (#2-h-\i) to[out=-90,in=\a] 
    coordinate (#2-hs-\i) (#2);
  \else
    \draw[->-] (#2) to[out=\a,in=-90]
    coordinate (#2-hs-\i) (#2-h-\i);
  \fi
}}

\newcommand{\tev}[2][0,0]{%
  \draw[->-] (#1) 
  .. controls ($(#1)-(0,.5)$) 
     and ($(#2)-(0,.5)$) .. (#2);}

\newcommand{\tcoev}[2][0,0]{%
  \draw[->-] (#1) 
  .. controls ($(#1)+(0,.5)$) 
     and ($(#2)+(0,.5)$) .. (#2);}


    \DeclareMathOperator{\car}{car}
\DeclareMathOperator{\id}{id}
\DeclareMathOperator{\tr}{tr}
\DeclareMathOperator{\Hom}{Hom}
\newcommand\comp{\mathrel{ \triangleright }}
\newcommand{\ev}[1]{\varepsilon_{\scriptscriptstyle #1}}
\newcommand{\coev}[1]{\eta_{\scriptscriptstyle #1}}
\newcommand{\ass}[3]{\alpha_{\scriptscriptstyle #1,#2,#3}}
\newcommand{\ot}{\otimes}
\newcommand{\R}{\mathbb{R}}
\newcommand{\F}{\mathbb{F}}
\newcommand{\Cx}{\mathbb{C}}
\newcommand{\cat}{\mathcal}
\newcommand{\bcat}{\mathrm}
\newcommand{\HomF}{\Hom_\F}
\newcommand{\Ob}{\mathrm{Ob}}
\newcommand{\Mor}{\mathrm{Mor}}
\newcommand{\dom}{\mathrm{dom}}
\newcommand{\cod}{\mathrm{cod}}
\newcommand{\C}{\cat{C}}
\newcommand{\ObC}{\Ob_\cat{C}}
\newcommand{\MorC}{\Mor_\cat{C}}
\newcommand{\Vect}{\bcat{Vect}}
\newcommand{\VectF}{\Vect_\F}
\newcommand{\bk}[1]{\langle #1 \rangle}


\newtheorem{proposition}{Proposição}
\newtheorem{corolary}{Corolário}
\newtheorem{lemma}{Lema}
\newtheorem{theorem}{Teorema}
\newtheorem*{remark}{Observação}
\theoremstyle{definition}
\newtheorem{example}{Exemplo}
